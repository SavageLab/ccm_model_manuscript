\documentclass[]{article}
\usepackage{amsmath}
\usepackage{amsfonts}
\usepackage{amssymb}
\usepackage{graphicx}
\usepackage[usenames, dvipsnames]{color}

%opening
\title{pH determines the energetic efficiency of the cyanobacterial CO2 concentrating mechanism: mathematical supplement}
\author{
  \small{Niall Mangan},
  \small{Avi Flamholz},
  \small{Rachel Hood}, 
  \small{Ron Milo} \&
  \small{David Savage}
}
\begin{document}
\date{}

\maketitle

\section{Equations when RuBisCO is saturated}
The analytic solution for the ${\rm CO_2}$ and ${\rm HCO_3^-}$ concentration in the carboxysome when RuBisCO is saturated is:

\begin{eqnarray}
C_{carboxysome} = \frac{N}{M} - \frac{R_c^3 V_{max}P}{3 M D}\\
H_{carboxysome} = K_{eq}(pH) C_{carboxysome}
\end{eqnarray}
where,

\begin{eqnarray}
N = (j_c+k_m^{eff}(pH_{out}))H_{out}((k_m^C+\alpha)G^C+\frac{D}{R_b^2}) + k_m^C C_{out} (k_m^{eff} G^H +\alpha G^C+\frac{D}{R_b^2}) \\
M = K_{eq}*k_m^{eff}\left((\alpha+ k_m^C)G^C +  \frac{D}{R_b^2}\right)
+ k_m^C\left(k_m^{eff} G^H + \frac{D}{R_b^2}\right) + \alpha k_m^{eff} G^H \\
P = ((\alpha + k_m^C)G^C+\frac{D}{R_b^2})(k_m^{eff} G^H + \frac{D}{R_b^2}) \\
G^C = \frac{D}{R_c^2 k_c^C} + \frac{1}{R_c}-\frac{1}{R_b} \\
G^H = \frac{D}{R_c^2 k_c^H} + \frac{1}{R_c}-\frac{1}{R_b} 
\end{eqnarray}
The derivation of this equation can be found in the supplementary material of (Mangan \& Brenner, eLife 2014). Here we have made a few modifications: (1) kept track of the carboxysome permeability to ${\rm CO_2}$, $k_c^C$, and ${\rm HCO_3^-}$, $k_c^H$, independently, (2) substituted the pH dependent equilibrium constant for the carbonic anhydrase reaction, $K_{eq}(ph) = \frac{V_{ca} K_{ba}}{V_{ba}K_{ca}}$, (3) written the ${\rm CO_2} \rightarrow {\rm HCO_3^-}$ reaction with $\alpha$ as the linear reaction rate (in Mangan 2014 the linear rate was $\alpha/K_\alpha$), (4) we have replaced the membrane permeability to ${\rm HCO_3^-}$ with the effective membrane permeability to the bicarbonate pool, and designated when is dependent on the external pH, $k_m^{eff}(pH_{out})$. This term only appears once in equation 3 for $N$. For all other $k_m^{eff} = k_m^{eff}(pH_{in})$ values it is dependent on the pH inside the cell, so we have dropped indicating the pH dependence to simplify the formulas.

\section{Analysis of membrane permeability effects}
\subsection{Cell permeabilty compared to diffusive velocities}
Examining equation (6) we note that for large carboxysome permeability $1/R_c$ will be the dominant term, and for smaller carboxysome permeability values the first term will be larger and dominate. Therefore $G^C \geq 1/R_c$. Studying the equations (3-7) we note that the terms $((\alpha + k_m^C)G^C+\frac{D}{R_b^2})$ appears repeatedly. We use the following argument:
\begin{eqnarray}
(\alpha + k_m^C)G^C \geq (\alpha + k_m^C)/R_c>>D/R_b^2, \\
\; {\rm if} \; (\alpha +k_m^C)>> D R_c/R_b^2
\end{eqnarray}

For even a small 20 nm diameter ($R_c = 10^{-6}$ cm) carboxysome this will hold as $k_m^C \approx 0.3 $ cm/s and $D R_c/R_b^2 = 4 \times 10^{-3}$ cm/s from the values in Table S1. So the membrane permeability to ${\rm CO_2}$ could be an order of magnitude too high in our model and this would still be a reasonable assumption. Therefore we will substitute 
\begin{equation}
(\alpha + k_m^C)G^C + D/R_b^2 \approx (\alpha + k_m^C)G^C.
\end{equation}

Inserting this into equations (1-5) we get
\begin{multline}
C_{carboxysome} = \frac{(j_c+k_m^{eff}(pH_{out}))H_{out}(k_m^C+\alpha)G^C 
	+ k_m^C C_{out} (k_m^{eff} G^H +\alpha G^C+\frac{D}{R_b^2})}
{K_{eq} k_m^{eff}(\alpha+ k_m^C)G^C 
	 + k_m^C\left(k_m^{eff} G^H + \frac{D}{R_b^2}\right) + \alpha k_m^{eff} G^H} \\
-\frac{R_c^3 V_{max}(\alpha + k_m^C)G^C(k_m^{eff} G^H + \frac{D}{R_b^2})/(3D)}
{ K_{eq} k_m^{eff} \left(\alpha+ k_m^C \right)G^C 
	+ k_m^C\left(k_m^{eff} G^H + \frac{D}{R_b^2}\right) + \alpha k_m^{eff} G^H}.
\end{multline}
We can divide through by  $(k_m^C+\alpha)$ to obtain:
\begin{multline}
	C_{carboxysome} = \frac{(j_c+k_m^{eff}(pH_{out}))H_{out}G^C + \frac{k_m^C}{(k_m^C+\alpha)} C_{out} (k_m^{eff} G^H +\alpha G^C+\frac{D}{R_b^2})}
	{K_{eq} k_m^{eff} G^C  
		+\frac{k_m^C}{(k_m^C+\alpha)} \left(k_m^{eff} G^H + \frac{D}{R_b^2}\right) 
		+ \frac{\alpha}{(k_m^C+\alpha)}  k_m^{eff} G^H} \\
	-\frac{R_c^3 V_{max}G^C(k_m^{eff} G^H + \frac{D}{R_b^2})/(3D)}
	{ K_{eq} k_m^{eff} G^C + \frac{k_m^C}{(k_m^C+\alpha)}\left(k_m^{eff} G^H + \frac{D}{R_b^2}\right) 
		+ \frac{\alpha}{(k_m^C+\alpha)} k_m^{eff} G^H}.
\end{multline}

We now want to examine the remaining terms in the membrane permeability to ${\rm CO_2}$, $k_m^C$.

\subsection{Membrane permeability to ${\rm CO_2}$ has little effect.}
There are two parameter groupings in equation (12) containing $k_m^C$:
\begin{eqnarray}
\frac{k_m^C}{k_m^C + \alpha}\\
\frac{\alpha}{k_m^C + \alpha}
\end{eqnarray}

Therefore if $k_m^C > \alpha$ or ${\rm CO_2}\rightarrow {\rm HCO_3^-}$ conversion is negligible the first term (13) reduces to 1, and the second reduces to $1/k_m^C$. We will return to the case where this conversion is not negligible later. 

With these two simplifications we obtain: 
\begin{multline}
	C_{carboxysome} = \frac{(j_c+k_m^{eff}(pH_{out}))H_{out}G^C +  C_{out} (k_m^{eff} G^H +\alpha G^C+\frac{D}{R_b^2})}
	{K_{eq} k_m^{eff} G^C  
		+ \left(k_m^{eff} G^H + \frac{D}{R_b^2}\right) 
		+ \frac{1}{k_m^C}  k_m^{eff} G^H} \\
	-\frac{R_c^3 V_{max}G^C(k_m^{eff} G^H + \frac{D}{R_b^2})/(3D)}
	{ K_{eq} k_m^{eff} G^C + \left(k_m^{eff} G^H + \frac{D}{R_b^2}\right) 
		+ \frac{1}{k_m^C} k_m^{eff} G^H}.
\end{multline}

Examining equation (15), note that the only appearance of the membrane permeability to ${\rm CO_2}$ is now in the denominator which we can rewrite as $  k_m^{eff}(G^C K_{eq}+ \frac{G^H}{k_m^C})  + \left(k_m^{eff} G^H + \frac{D}{R_b^2}\right) $. Using this equation, we can write a strong bound on when the membrane permeability will effect the function of the CCM. 

We find $k_m^C$ has no significant effect when $K_{eq}G^C > \frac{G^H}{k_m^C}$ or $ k_m^C > \frac{G^H}{G^C K_{eq}}$. If we assume that the carboxysome permeability to ${\rm CO_2}$ will always be smaller than or equal to the permeability to ${\rm HCO_3^-}$ ($k_c^C\geq k_c^H$) then $G^H \geq G^C$ and $\frac{G^H}{G^C} \leq 1$, so $k_m^C$ will be negligible as long as $k_m^C > 1/K_{eq}$. For pH $> 6.6$, $1/K_{eq} > 0.3$ and therefore the assumed value of $k_m^C=0.3$ will be negligible. However, if the cell operated in a lower pH regime and the membrane permeability was substantially lower to $CO_2$ it would begin to effect the ${\rm CO_2}$ concentration. 

Thus far we have made a series of observations about the size of terms compared to the membrane permeability to ${\rm CO_2}$ and found that when $(\alpha +k_m^C)>> D R_c/R_b^2$, $k_m^C > \alpha$ and $ k_m^C > \frac{G^H}{G^C K_{eq}} \approx 1/K_{eq}$ the ${\rm CO_2}$ concentration in the carboxysome reduces to 
\begin{multline}
C_{carboxysome} = \frac{(j_c+k_m^{eff}(pH_{out}))H_{out}G^C +  C_{out} (k_m^{eff} G^H +\alpha G^C+\frac{D}{R_b^2})}
	{k_m^{eff}(G^C K_{eq} +  G^H) + \frac{D}{R_b^2} }\\
-\frac{R_c^3 V_{max}G^C(k_m^{eff} G^H + \frac{D}{R_b^2})/(3D)}
	{k_m^{eff}(G^C K_{eq} +  G^H) + \frac{D}{R_b^2} }.
\end{multline}

We can make a similar argument taking the equation for the ${\rm CO_2}$ concentration at the cell membrane:
\begin{multline}
C_{cytosol}(r = R_b) = \frac{k_m^C C_{out} - (\alpha + k_m^C)C_{carboxysome}}{(\alpha+k_m^C)G^C + D/R_b^2}G^C + C_{carboxysome}\\
\approx C_{out}
\end{multline}

This means that the ${\rm CO_2}$ leakage term will be negligible since the cytosolic ${\rm CO_2}$ concentration will be approximately equal to the external ${\rm CO_2}$ concentration. The ${\rm HCO_3^-}$ transport required to sustain a given internal inorgainc carbon pool will then be:

\begin{multline}
j_c H_{out} =  \left(\frac{R_c^3}{3 R_b^2}  V_{max}  -k_m^C \left( C_{out} - C_{cytosol} \right) - k_m^{eff} H_{out} + k_m^{eff} H_{cytosol} \right) \\ 
\; = \left(\frac{R_c^3}{3 R_b^2}  V_{max}  - k_m^{eff} H_{out} + k_m^{eff} H_{cytosol} \right)
\end{multline}

We can calculate $H_{carboxyome} = K_{eq}C_{carboxysome}$ from equation (17), and is therefore also independent of $k_m^C$. In previous work we showed that 
\begin{equation}
H_{cytosol} =  \frac{(j_c + k_m^{eff}(pH_{out}))H_{out} + \frac{\alpha}{K_\alpha} C_{cytosol}(r=R_b) - k_m^{eff} H_{carboxysome} }{k_m^{eff} G^H + \frac{D}{R_b^2}}G^H
\end{equation}

We have now shown that all the terms in $H_{cyto}$ are negligibly dependent on the membrane permeability to ${\rm CO_2}$. Therefore, the ${\rm HCO_3^-}$ transport level require to satisfy equation (18) is independent of the membrane permeability to ${\rm CO_2}$. This observation is consistent with the low flux of ${\rm CO_2}$ leakage in main text Figure 2.

\subsection{Without facilitated ${\rm CO_2}$ uptake external ${\rm CO_2}$ has little effect}

Unless conversion from ${\rm CO_2}$ to ${\rm HCO_3^-}$ is large we note that the second $C_{out}$ term in equation(15) is negligible for the regimes we study. We will revisit ${\rm CO_2}$ uptake and recycling later. Comparing this term against the first term in the numerator, again allows us to put a quantitative description on when this regime holds. Additionally we find that when the transport of ${\rm HCO_3^-}$ is significant ($j_c > k_m^{eff}(pH_{out})$) we arrive at

\begin{eqnarray}
	C_{carboxysome} = \frac{j_c H_{out}G^C - R_c^3 V_{max}G^C(k_m^{eff} G^H + \frac{D}{R_b^2})/(3D)}
	{k_m^{eff}(G^C K_{eq} +  G^H) + \frac{D}{R_b^2} }\\
	H_{carboxysome} = K_{eq} C_{carboxysome}
\end{eqnarray}

\section{Effect of Carboxysome permeability}

Recalling the equation for $G^C = \frac{D}{R_c^2 k_c^C} + \frac{1}{R_c}-\frac{1}{R_b}$, we can see that the carboxysome permeability to  ${\rm CO_2}$ will only matter if $ \frac{D}{R_c^2 k_c^C} >\frac{1}{R_c}$. In other words the carboxysome permeability to ${\rm CO_2}$, $k_c^C$, begins to effectively trap ${\rm CO_2}$ in the carboxysome when $k_c^C < \frac{D}{R_c} \approx 2$ cm/s for our base case of a 100 nm carboxysome ($R_c = 50$ nm). Similarly $G^H \approx \frac{D}{R_c^2 k_c^H}$ when $k_c^H < \frac{D}{R_c}$. As common thinking is that $k_c^H \geq k_c^C$,  $k_c^H < \frac{D}{R_c}$ may not always hold when  $k_c^C < \frac{D}{R_c}$. 

\subsubsection{Different carboxysome peremability for ${\rm HCO_3^-}$}
An existing hypothesis in the CCM literature is that the carboxysome has differential permeability and is more permeable to ${\rm HCO_3^-}$ and less permeable to ${\rm CO_2}$. Intuitively this would allow more ${\rm HCO_3^-}$ into the carboxysome and trap more ${CO_2}$, thereby accumulating more inorganic carbon in the form of ${\rm CO_2}$. We use our model to test weather differential carboxysome permeability enables higher carboxysomal ${\rm CO_2}$ concentration for the same level of ${\rm HCO_3^-}$ transport. In the Figure S12 we show the $k_c$ vs $j_c$ phase space where we have plotted the carboxysome permeability to ${\rm CO_2}$, $k_c^C$, on the y-axis. We plot different ratios (1, 10, 100, 1000) between $k_c^C$ and the carboxysome permeability to ${\rm HCO_3^-}$, $k_c^H = {\rm ratio} \times k_c^C$. 

Examining Figure S12, we see that making the carboxysome more permeable to ${\rm HCO_3^-}$ does not improve the function of the CCM as drastically as on might assume. The "turn on" of ${\rm CO_2}$ accumulation with decreasing permeability is unaffected by changes to $k_c^H$, and depends only on the permeability ${\rm CO_2}$, $k_c^C$.The "turn off" of accumulation for lower carboxysome permeabilities is greatly effected by the permeability of the carboxysome to ${\rm HCO_3^-}$, $k_c^H$. These two effects are exactly what we previously discussed as defining the carboxysome permeability optimum.

As we start at the top of the y-axis and decrease the carboxysome permeability the following occurs: At high permeability not enough ${\rm CO_2}$ is trapped, but ${\rm HCO_3^-}$ enters readily. As we moved to lower permeabilities ${\rm CO_2}$begins to be trapped, but there is a window where ${\rm HCO_3^-}$ still enters enough to supply the system. Eventually the carboxysome begins to restrict ${\rm HCO_3^-}$ entry. If the carboxysome is more permeable to ${\rm HCO_3^-}$ than to ${\rm CO_2}$ then the window where ${\rm CO_2}$ trapping is effective without restricting ${\rm HCO_3^-}$ entry broadens. The width of this window (on the y-axis) will also depend strongly on how much of the ${\rm CO_2}$ is being fixed.

The "turn off" of the optimum, caused by not allowing enough ${\rm HCO_3^-}$ into the carboxysome, does slightly increase the amount of transport required to saturate RuBisCO at the carboxysome optimum. The reduction in transport required, and therefore CCM cost is around 5\% when going from a $k_c^C$ to $k_c^H$ ratio of 1 to 1000.


%Not sure if we need analysis of the solution (also haven't checked this version carefully):
%\begin{equation}
%C_{carboxysome} = \frac{j_c H_{out}\frac{D}{R_c^2 k_c^C}- R_c^3 V_{max}\frac{D}{R_c^2 k_c^C}(k_m^{eff} \frac{D}{R_c^2 k_c^H}+ \frac{D}{R_b^2})/(3D)}
%{k_m^{eff}(\frac{D}{R_c^2 k_c^C} K_{eq} +  \frac{D}{R_c^2 k_c^H}) + \frac{D}{R_b^2} }
%\end{equation}
%
%dividing numerator and denominator through by $\frac{D}{R_c^2 k_c^C}$
%\begin{equation}
%C_{carboxysome} = \frac{j_c H_{out}- R_c^3 V_{max}(k_m^{eff} \frac{1}{R_c^2 k_c^H}+ \frac{1}{R_b^2})/3}
%{k_m^{eff}(K_{eq} +  \frac{k_c^C}{k_c^H}) + \frac{k_c^C R_c^2}{R_b^2} }
%\end{equation}

\section{Effect of membrane permeability to ${\rm H_2CO_3}$}

The sensitivity of the cost to our assumption for the value of the membrane permeability to  ${\rm H_2CO_3}$ can be determined from the equation derived previously. If we are in a regime where ${\rm CO_2}$ leakage is negligible, as is the regime presented in the main paper, the second line holds. 
\begin{multline}
j_c H_{out} =  \left(\frac{R_c^3}{3 R_b^2}  V_{max}  -k_m^C \left( C_{out} - C_{cytosol} \right) -k_m^{eff}(pH_{out}) H_{out} +k_m^{eff}(pH_{in})  H_{cytosol} \right) \\ 
\; = \left(\frac{R_c^3}{3 R_b^2}  V_{max}  - k_m^{eff}(pH_{out}) H_{out} + k_m^{eff}(pH_{in}) H_{cytosol} \right)
\end{multline}
In this equation $k_m^{eff} = k_m^{H_2CO_3} \times 10^{(pK_1 - pH)}$. Therefore, the leakage of ${H_{total}}$ out of the cell will depend linearly on what we assume for $ k_m^{H_2CO_3}$. This linear dependence is past on to the active ${\rm HCO_3^-}$ transport required to replenish the leaked inorganic carbon, and therefore onto the CCM cost. In Figure S14 you can see this effect, where going from $ k_m^{H_2CO_3} = 3\times10^{-2}$ to $ k_m^{H_2CO_3} = 3\times10^{-3}$ (an order of magnitude change), decreases the active ${\rm HCO_3^-}$ transport needed by an order of magnitude. Decreasing  to $ k_m^{H_2CO_3} = 3\times10^{-4}$ is a little less than an order of magnitude, indicating that the linear dependence breaks down and ${\rm CO_2}$ leakage would become important for that value. There is also an order of magnitude change in the optimal carboxysome permeability from ${\rm 10^{-4}}$ to ${\rm 10^{-5}}$ across the 2 order of magnitude change in $ k_m^{H_2CO_3}$ we are checking.

%We can define the \% of ${\rm CO_2}$ scavanged compared to that leaked out of the carboxysome as:
%\begin{equation}
%S_\% = \frac{k_c(C_{carboxysome} - C_{cytosol}(R_c))}{k_m^C(C_{cytosol}(R_b)-C_{out})}
%\end{equation}
%Our analytically derived function for the cytosolic ${\rm CO_2}$ concentration as a function of the radius:
%\begin{equation}
%C_{cytosol} = \frac{k_m^C C_{out} - (\alpha-k_m^C) C_{carboxysome}}{(\alpha+k_m^C)G^C +\frac{D}{R_b^2}}\left(\frac{D}{k_c^C R_c^2} + \frac{1}{R_c} - \frac{1}{r}\right) +C_{carboxysome}
%\end{equation}
%We can use equation (24) in equation (23) and rearrange equation (23) to find the concentration of ${\rm CO_2}$ in the carboxysome resulting in a particular scavenging percentage. This can then be plotted on the ${\rm HCO_3^-}$ transport vs carboxysome permeability plot to see where the functionality is important.
%\begin{equation}
%C_{carboxysome}(S_\%) = \frac{(k_m^C S_\% ((\alpha + k_m^C)G^C + \frac{D}{R_b^2})- k_m^C(k_m^C S_\% G^C +\frac{D}{R_c^2}))C_{out}}{k_m^C S_\% ((\alpha + k_m^C)G^C + \frac{D}{R_b^2})- (\alpha+k_m^C)(k_m^C S_\% G^C +\frac{D}{R_c^2})}
%\end{equation}

%\begin{equation}
%C_{csome}(C_{cytosol}(R_b)= C_{out}) = C_{out} (1+(\alpha + k_m^C)G^C \frac{R_b^2}{D} - \frac{k_m^C R_b^2}{D})
%\end{equation}



\end{document}